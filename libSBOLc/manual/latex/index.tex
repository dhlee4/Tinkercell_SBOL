\hypertarget{index_overview}{}\section{Overview}\label{index_overview}
There are a few different top-\/level subdirectories in the lib\+S\+B\+O\+Lc folder\+:


\begin{DoxyItemize}
\item \href{https://github.com/SynBioDex/libSBOLc/tree/master/schema}{\tt schema} holds the S\+B\+O\+L schema files copied from lib\+S\+B\+O\+Lj
\item \href{https://github.com/SynBioDex/libSBOLc/tree/master/source}{\tt source} contains source code for the library
\item \href{https://github.com/SynBioDex/libSBOLc/tree/master/examples}{\tt examples} contains example code that uses the library, as well as the xml test files from lib\+S\+B\+O\+Lj
\item \href{https://github.com/SynBioDex/libSBOLc/tree/master/manual}{\tt manual} contains source code for generating the online documentation
\item \href{https://github.com/SynBioDex/libSBOLc/tree/master/tests}{\tt tests} contains unit tests, as well as tests that verify correct handling of the example xml files
\item \href{https://github.com/SynBioDex/libSBOLc/tree/master/wrapper}{\tt wrapper} contains code for generating the Python wrapper using S\+W\+I\+G
\item Finally, there's the build folder for temporary build files, and the release folder for finished binaries. They're created by C\+Make.
\end{DoxyItemize}

When running C\+Make and compiling, the main sbol library is always created first because most of the optional targets depend on it\+: 

Once built, the executables just depend on finding the correct xml files\+: \hypertarget{index_tour}{}\section{A brief tour of the source code}\label{index_tour}
There are a bunch of C source files, but they fall into a few general categories\+:


\begin{DoxyItemize}
\item dnasequence.\+h, sequenceannotation.\+h, dnacomponent.\+h, and collection.\+h define the S\+B\+O\+L Core objects.
\item array.\+h, object.\+h, and property.\+h define building blocks for those objects.
\item document.\+h defines an S\+B\+O\+L Document.
\item validator.\+h, reader.\+h, and writer.\+h perform operations on documents. They each have one main algorithm, plus maybe some non-\/exported supporting functions.
\item Then there are a few other miscellaneous files like prototypes.\+h, constants.\+h.\+in, and utilities.\+h; they each do something unique but limited.
\end{DoxyItemize}

You might be confused and/or annoyed with the profusion of similarly-\/named structs and functions relating to S\+B\+O\+L objects. I know I would be seeing it for the first time. Here are a couple diagrams to clarify what they're supposed to represent\+:



\hypertarget{index_doxygen}{}\section{Updating this documentation}\label{index_doxygen}
The code for this page is in mainpage.\+dox; the rest of the documentation is scattered throughout the source/$\ast$.h files. There are also a couple graphs, which can be edited with the y\+Ed graph editor from \href{http://www.yworks.com/en/products_yed_about.html}{\tt http\+://www.\+yworks.\+com/en/products\+\_\+yed\+\_\+about.\+html}

After editing the source files, you also need to generate html using Doxygen, and upload it to Git\+Hub.

First, make sure you have Doxygen installed. Then check {\ttfamily S\+B\+O\+L\+\_\+\+G\+E\+N\+E\+R\+A\+T\+E\+\_\+\+M\+A\+N\+U\+A\+L} in C\+Make, {\ttfamily cd} into {\ttfamily lib\+S\+B\+O\+Lc/build}, and run {\ttfamily make}. Doxygen will generate a {\ttfamily lib\+S\+B\+O\+Lc/release/manual/html} folder full of html files. To change settings, edit Doxyfile.\+in first.

Copy the H\+T\+M\+L to somewhere outside the main lib\+S\+B\+O\+Lc directory, then {\ttfamily git checkout gh-\/pages}. It's a special branch with nothing but H\+T\+M\+L files. There may also be some leftover untracked files from the master branch. Delete everything in there, and replace it with the newly generated html files. Then {\ttfamily git push origin gh-\/pages}. A minute or so later, the new documentation should be available at \href{http://SynBioDex.github.com/libSBOLc/}{\tt http\+://\+Syn\+Bio\+Dex.\+github.\+com/lib\+S\+B\+O\+Lc/}\hypertarget{index_swig}{}\section{Updating the Python wrapper}\label{index_swig}
To update the low-\/level Python wrapper, all you need to do is\+:


\begin{DoxyItemize}
\item check S\+B\+O\+L\+\_\+\+G\+E\+N\+E\+R\+A\+T\+E\+\_\+\+P\+Y\+T\+H\+O\+N in C\+Make
\item build the project
\item keep \+\_\+libsbol.\+so and libsbol.\+py from the lib\+S\+B\+O\+Lc/release/wrapper folder
\end{DoxyItemize}

These are generated by S\+W\+I\+G, and represent a direct translation of the C functions into Python. They're cross-\/platform so you should only need to generate them once.

There is also a high-\/level wrapper that maps the C-\/like S\+W\+I\+G functions into a more natural Python interface. It's kept in a separate repository at \href{https://github.com/SynBioDex/libSBOLpy,}{\tt https\+://github.\+com/\+Syn\+Bio\+Dex/lib\+S\+B\+O\+Lpy,} and will need to be manually adjusted when new functions are added to the S\+W\+I\+G wrapper.

The low-\/level wrapper also works by itself though. From the Python interpreter, import libsbol to use it. While the high-\/level interface is more pythonic, the low-\/level one is more useful for prototyping algorithms that will be re-\/written in C.

One other thing to remember is that although the generated files are cross-\/platform, they only work with the version of Python they were compiled against. If you try to import the libsbol module from a different version you may get a seeminly unrelated error like \char`\"{}\+Fatal Python error\+: Py\+Thread\+State\+\_\+\+Get\+: no current thread\char`\"{}. 